\chapter{Results}
\label{res}

\section{Methodology}
To determine the performance of the Parallel-J solutions to the problems listed in Chapter \ref{probs}, 
each solution was run on varying problem sizes using 1, 2, and 4 threads. 
For comparison, the performance of the solutions in 
C with OpenMP for the same problems are also listed. 
These C solutions were run on larger problem sizes 
in order to demonstrate their scaling more effectively 
(see Section \ref{discres} for more details).
The Scala API Benchmark object was used to measure the performance of the Parallel-J solutions, 
which allowed for ease of use from the command line, automatic timing, 
and forcing the Java garbage collector to run between program runs. 
For the C programs, timing had to be done manually for each program. 
For all program results, the run time is given in seconds.

All solutions were run (and in the case of the C solutions, compiled) on the Dione00 work station. %TODO get machine specs! 
For the Parallel-J solutions (all of which can be found in Appendix \ref{apsol}), % TODO list benchmark, refactor
a single trial consisted of running each program 
on a specific problem size with a specific number of threads 
a total of 6 times and measuring their execution time. 
the results of the latter 5 program runs were averaged. 
The first program run was omitted in order to account for 
the Java Just-In-Time (JIT) compiler making optimizations on the running Parallel-J code.
For the solutions in C with OpenMP,
a single trial consisted of running each program with a specific number of threads a total of 5 times. 
The performance for all the program runs for the C trials were averaged.

\section{Calculating Pi using Numerical Integration}
\label{respi}
The number of operations performed in this problem scales 
as the number of divisions $n$ between the interval $[0,1]$ grows. 
For our tests, we measured performance for $n=10,000$, $n=20,000$, ..., $n=100,000$.
Table \ref{tnip} below gives the outcomes.

\begin{table}[htbp]
\begin{tabular}{|r||c|c|c||c|c|c||c|c|c||c|c|c||}
	\hline
		& \multicolumn{3}{|c|}{$n=10,000$} & \multicolumn{3}{|c|}{$n=20,000$} & \multicolumn{3}{|c|}{$n=30,000$} & \multicolumn{3}{|c|}{$n=40,000$} \\
	\hline
		Threads & 1 & 2 & 4 & 1 & 2 & 4 & 1 & 2 & 4 & 1 & 2 & 4 \\
	\hline
	\hline
		Parallel-J & 10.1 & 10.1 & 10.2 & 31.5 & 38.4 & 38.5 & 70.6 & 90.8 & 91.0 & 126.2 & 152.5& 152.8 \\
	\hline
\end{tabular}

\begin{tabular}{|r|c|c|c||c|c|c||c|c|c||}
	\hline
		& \multicolumn{3}{|c||}{$n=50,000$} & \multicolumn{3}{|c|}{$n=60,000$} & \multicolumn{3}{|c|}{$n=70,000$} \\
	\hline
		Threads & 1 & 2 & 4 & 1 & 2 & 4 & 1 & 2 & 4 \\
	\hline
	\hline
		Parallel-J & 173.7 & 236.0 & 235.6 & 251.0 & 361.0 & 361.0 & 342.2 & 490.2 & 491.3 \\
	\hline
\end{tabular}

\begin{tabular}{|r|c|c|c||c|c|c||c|c|c||}
	\hline
		 & \multicolumn{3}{|c|}{$n=80,000$} & \multicolumn{3}{|c|}{$n=90,000$} & \multicolumn{3}{|c|}{$n=100,000$} \\
	\hline
		Threads & 1 & 2 & 4 & 1 & 2 & 4 & 1 & 2 & 4 \\
	\hline
	\hline
		Parallel-J & 651.6 & 653.7 & 653.3 & 826.1 & 769.3 & 814.5 & 1017.5 & 1020.0 & 1021.5 \\
	\hline
\end{tabular}
\caption{Performance results for calculating $\pi$}
\label{tnip}
\end{table}

\section{Game of Life}
For the two-dimensional version of the ``Game of Life'', 
the number of operations performed in this problem scales 
as the dimensions $x$ and $y$ grow larger, 
and also as the number of desired iterations $i$ grows. 
The latter, however, is fundamentally sequential, 
so $i$ was fixed to be relatively small with value 10.
In order to simplify the benchmarking process, 
trials were only run on square boards ($x=y$). 
Table \ref{tgol} below gives the outcomes for $x$ and $y$ 
ranging from 100 to 300 by increments of 25.

\begin{table}[htbp]
\begin{tabular}{|r||c|c|c||c|c|c||c|c|c||}
	\hline
		& \multicolumn{3}{|c|}{$x=100$} & \multicolumn{3}{|c|}{$x=125$} & \multicolumn{3}{|c|}{$x=150$} \\
	\hline
		Threads & 1 & 2 & 4 & 1 & 2 & 4 & 1 & 2 & 4 \\
	\hline
	\hline
		Parallel-J & 66.4 & 66.5 & 66.4 & 159.8 & 159.6 & 159.9 & 326.8 & 335.5 & 348.1 \\
	\hline
\end{tabular}
\begin{tabular}{|r||c|c|c||c|c|c||c|c|c||}
	\hline
		& \multicolumn{3}{|c|}{$x=175$} & \multicolumn{3}{|c|}{$x=200$} & \multicolumn{3}{|c|}{$x=225$} \\
	\hline
		Threads & 1 & 2 & 4 & 1 & 2 & 4 & 1 & 2 & 4 \\
	\hline
	\hline
		Parallel-J & 725.8 & 728.3 & & & & & & & \\
	\hline
\end{tabular}
\begin{tabular}{|r||c|c|c||c|c|c||c|c|c||}
	\hline
		& \multicolumn{3}{|c|}{$x=250$} & \multicolumn{3}{|c|}{$x=275$} & \multicolumn{3}{|c|}{$x=300$} \\
	\hline
		Threads & 1 & 2 & 4 & 1 & 2 & 4 & 1 & 2 & 4 \\
	\hline
	\hline
		Parallel-J & & & & & & & & & \\
	\hline
\end{tabular}
\caption{Performance results for the Game of Life}
\label{tgol}
\end{table}

\section{Merge Sort}
The number of operations required for this problem 
grows as the number $n$ of values to sort increases. 
Because of the problem constraint listed in Section \ref{mgdes}, 
problem sizes of the form $2^n$ were chosen, 
where $14 \le n \le 16$. 
The reader should note that due to time constraints, 
the Parallel-J version of this solution uses a \texttt{merge} function 
implemented imperatively in Scala, rather than 
implementing it in an equivalent way to the J idiom given in Appendix \ref{jmfull}.
Table \ref{tmrg} below gives the outcomes.

\begin{table}[htbp]
\begin{tabular}{|r||c|c|c||c|c|c||c|c|c||}
	\hline
		& \multicolumn{3}{|c|}{$n=2^{14}$} & \multicolumn{3}{|c|}{$n=2^{15}$}& \multicolumn{3}{|c|}{$n=2^{16}$} \\
	\hline
		Threads & 1 & 2 & 4 & 1 & 2 & 4 & 1 & 2 & 4 \\
	\hline
	\hline
		Parallel-J & 2.9 & 2.5 & 2.5 & 10.6 & 9.2 & 9.2 & 41.1 & 35.5 & 35.7 \\
	\hline
\end{tabular}
\caption{Performance results for merge sort}
\label{tmrg}
\end{table}
