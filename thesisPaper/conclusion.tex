\chapter{Conclusion}
\label{conc}

\section{Future Research}

\subsection{Parallel Implementation of J}
This research could be useful for future work which provides 
a parallel alternative of J.
No matter the implementing language chosen, 
the considerations in Section \ref{desp} 
over which language features to include in a prototype, 
and the discussion in Section \ref{fridp} of the behavior and inherent data parallelism of the rank operator, 
would likely be useful to those not intimately familiar with the current J implementation.

Also to be taken into consideration should be 
the forces which lead to the choice of Scala as the language 
for this research, also listed in Section \ref{desp}. 
If future work is done in this area in Scala 
or another language with suitable programming paradigms and libraries 
for developing a parallel implementation of J, 
then this research should be used as a prototype 
to help guide development.

Alternatively, a future parallel implementation of J
could be done in the C programming language. 
In order to begin this approach, 
it is strongly recommended that researchers first 
familiarize themselves with the working of the current implementation of J\cite{ioj}, 
the source code and documentation of which 
is available freely under open source licenses.
In addition, it seems that there are at least two viable options in such an approach 
which are possibly not mutually exclusive.
Future researchers may use shared memory parallel environments
such as the OpenMP library 
to parallelize operations within a single instance of a parallel J.

Also, the current implementation of J already includes 
functionality to allow for multiple instances of J to be running in the same process 
without race conditions.
Therefore, it may be possible to approach future work using 
a distributed memory parallel environment, such as MPI. % TODO cite!

\subsection{Sequential and Parallel Scala Library for Regular Arrays using Function Rank}
Another possible extension to the work presented here 
is the development of Scala library for arbitrary dimensional collections which uses function rank. 
This library would ideally support parallelism in much the same way 
Scala's current collections library does\cite{pc},
through conversions between sequential and parallel implementations of each collection type. 
However, it should be clear that even 
a purely sequential version of such a library would be useful 
for solving problems which requires operations on several different dimensions, 
or which are naturally expressed in a higher dimension than the original problem description.

At the time this research was conducted, 
the author was not aware of current work being done in generic, polytipic
